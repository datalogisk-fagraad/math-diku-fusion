\documentclass[a4paper]{article}

\usepackage[utf8]{inputenc}
\usepackage[danish]{babel}

\usepackage{mathpazo}


\usepackage{fancyhdr}%
\usepackage{lastpage}%
%
\pagestyle{fancy}%
\lhead{}%
\chead{}%
\rhead{}%
\cfoot{\thepage/\pageref{LastPage}}%
\renewcommand{\headrulewidth}{0in}%
%

\setlength{\parskip}{0.2cm}
\setlength{\parindent}{0cm}

%\usepackage{draftwatermark}
%\SetWatermarkText{DRAFT}
%\SetWatermarkScale{1}

\usepackage{listings}

\lstset{columns=flexible,basicstyle=\ttfamily}

\begin{document}

ÅBENT BREV

Dekan John Renner Hansen \\
Det Natur- og Biovidenskabelige Fakultet \\
Københavns Universitet

Fakultetsdirektionen ved SCIENCE planlægger at fusionere MATH og DIKU til ét
institut med virkning fra 1. januar 2015.

Undertegnede fagråd ved \mbox{SCIENCE} finder følgende forhold ved processen
dybt bekymrende:

\begin{itemize}

\item De studerende og ansatte ved de berørte institutter blev først informeret
om fusionsplanen den 15. september kl. 16, hvor der blev inviteret til et stormøde
om planen, den 16. september kl. 14.15. Mødet skulle være ``startskuddet på en
drøftelsesfase''\cite{invite}.

Mødetidspunktet lå hen over to skemagrupper i en travl undervisningsuge.  Der
indkaldtes til mødet med mindre end 24 timers varsel.

At informere om et helt uforventet forslag på denne made, forhindrer dialog. De
berørte parter har ikke tid til at diskutere indbyrdes og dele meninger,
hvilket nødvendigt for at de kan forholde sig til sagen som præsenteret.

\item Drøftelsesfasen skulle løbe fra den 15. september til den 20. oktober ---
en måned i en travl blok, uden en undervisningsfri uge.

% Der er tale om nedlæggelse af to internationalt anerkendte institutter, og
% opbygning af et helt nyt institut.

%Der skal stærke argumenter på bordet når administrative forhold risikerer at
%påvirke veletableret og anerkendte forskningsinstituter negativt. De berørte
%institutter skal også have rum til at bekræfte eller modspille argumenterne.
%Tidsfristerne under processen har generelt været uhensigtsmaessigt korte i
%forhold til forslagets omfang.


Der skal stærke argumenter på bordet når administrative forhold påvirker
veletableret og anerkendte forskningsinstitutter. De berørte institutter skal
have rum til at bekræfte eller modspille argumenterne.  Tidsfristerne under
hele processen har generelt været uhensigtsmaessigt korte i forhold til
forslagets omfang.

% Unaturligt
% Advertising
% Høringsprocess

% Det er bekymrende at administrative vanskelighed medfører beslutninger der
% påvirker fagligheden.

% Ikke vær for specifik om fagråderne og de udtalelser, de kommer ikke
% nødvendigvis.

\item Fakultetsdirektionen har ikke meldt ud om et decideret høringsprocedure.

Det er ikke blevet meldt ud hvordan, og hvorledes høringssvar kan angives.
Fristen, den 3.  november, er noget som er blevet uformelt kommunikeret iblandt
de berørte parter. Kommunikationen mellem fakultetsdirektionen og de studerende
har især været mangelfuld.

\item Fakultetsdirektionen har ikke gjort relevant dokumentation let
tilgængeligt for de berørte parter, hverken op til stormødet den 16.
september, eller siden.

Dette udviser en generel manglende gennemsigtighed, som forhindrer sund kritik
af ledelsens planer. Relevant dokumentation er, af sagens natur, ikke lige
tilgænglig for fakultetsdirektionen og andre.

\item Et tilsvarende forslag blev afvist i 2011. De allerede pressede
studerende og ansatte skal nu bruge (arbejds)tid og (pyskisk) overskud på at
gennemgå samme process kort efter igen. Det er bekymrende at Dekanen nægter at
tage de ``gamle argumenter'' i betragtning:

``Vi kender historikken. Vi kender alle argumenterne. Jeg har læst dem fra for
to og et halvt år siden. Vi vil gerne se en masse nye argumenter i den her
diskussion. (red. salen udbryder i latter.) Hvis det var sådan, at jeg, når jeg
læste argumenterne igennem, havde følt mig overbevist om at det her, det var
ikke bæredygtigt, så ville jeg ikke stå her i dag. Så det kan godt være i synes
det er morsomt at der skal nye argumenter på bordet...''\cite{stormoede} 

Bemærk, at grundet manglende dokumentation, var det uklart hvilke ``gamle
argumenter'' der er tale om.

\end{itemize}

Vi frygter for tilsvarende manglende indragelse af relevante parter, og
manglende gennemsigtighed, ved lignende problemstillinger i fremtiden.

I solidaritet med vores medstuderende, skal det desuden bemærkes, at vi finder
det problematisk at et af de berørte institutter (DIKU), er, i løbet af de
sidste 6 år, blevet splittet op over 4-5 geografiske lokationer. Dette
forringer forholdene for de studerende væsentligt, både fagligt og
studiemiljøsmæssigt, grundet den store (fysiske) afstand mellem de studerende
og ansatte, og de ansatte imellem.

% Vi støtter hermed vores medstuderendes ønske om, at DIKU samles som
% selvstændig institut på én addresse: fra de studerendes synspunkt, allerhelst
% igen omkring studenterkøkkenet på Universitetsparken 1, for at sikre et godt
% studiemiljø.  Derudover, opfordrer vi kraftigt til, at instituttet samles
% omgående, og ikke først i 2017. Dette er begrundet i de åbenlyse problemer
% den nuværende situation medfører.

\begin{flushright}

\footnotesize\sffamily\itshape

Datalogisk Fagråd

...

% Matematisk Fagråd

% Fysisk Fagråd

\end{flushright}

\begin{thebibliography}{9} % 9 if < 10 references, 99 if < 100 references, etc.

%\bibitem[Invitation til møde om fusion af MATH og DIKU]{invite}

\bibitem[1]{invite}

John Renner Hansen. \emph{Invitation til møde om fusion af MATH og DIKU}.

Studiebesked nummer 522 på KUnet. Oprettet den 15. september 2014 af Phillipp
Lorenzen, kl. 13.48. Sidst ændret den 19. september 2014 af Phillipp Lorenzen
kl. 12:49.

\begin{lstlisting}
https://intranet.ku.dk/science/dk/studerende/studiebeskeder/
Sider/Forms/DispForm.aspx?ID=522
\end{lstlisting}

%\bibitem[Videooptagelse fra stormødet den 16. september]{stormoede}

\bibitem[2]{stormoede}

John Renner Hansen. \emph{Debatmøde om eventuel sammenlægning af MATH og DIKU}.
Videooptagelse fra stormødet den 16. september. Der henvises bl.a. til
tidsstempel 26:54 og frem.

\begin{lstlisting}
https://ku.23video.com/1086107.ihtml/player.html?
token=f4df3b5b26504e57941ad44f08b3b954&photo_id=10321596
\end{lstlisting}

\end{thebibliography}

\end{document}

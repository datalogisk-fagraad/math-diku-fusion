\documentclass{article}
\usepackage{url}
\usepackage[utf8]{inputenc}

\usepackage{fancyhdr}%
\usepackage{lastpage}%
%
\pagestyle{fancy}%
\lhead{ÅBENT BREV}%
\chead{}%
\rhead{31. oktober 2014}%
\cfoot{\thepage/\pageref{LastPage}}%
\renewcommand{\headrulewidth}{0in}%
%

\begin{document}
\title{Vedrørende forholdene for DIKU-studerende i forbindelse med den planlagte MATH+DIKU fusion}
\date{}
\thispagestyle{fancy}

{\setlength{\parindent}{0pt} Dekan John Renner Hansen} \\
Det Natur- og Biovidenskabelige Fakultet \\
Københavns Universitet

{\let\newpage\relax\maketitle}

\thispagestyle{fancy}

\vspace{-25pt}

Vi, som medlemmer af Datalogisk Fagråd og Datalogisk Kantineforening, ønsker
hermed at udtrykke vores bekymringer omkring det foreslåede fusion af
Datalogisk Institut (DIKU) og Institut for Matematiske Fag (MATH).

Datalogisk Institut står igen overfor at blive flyttet i forbindelse med den
planlagte fusion. Studenterkøkkenet, drevet af frivillige medlemmer af
Datalogisk Kantineforening, er i centrum for DIKUs enormt levende sociale og
faglige miljø, og nu trues dets overlevelse endnu engang.

Beslutninger på fakultetsniveau har, igennem de seneste år, haft meget negative
konsekvenser for vores studiemiljø generelt. Vi har derfor svært ved at have et
optimistisk syn på den fremlagte fusionsplan, uafhængigt af alle andre
argumenter.

\section{Studenterkøkkenet}

Datalogisk Kantineforeningens medlemmer udgør samtlige DIKU-studerende og
ansatte\footnote{Kantineforeningens vedtægter:
\url{http://kantinen.org/_media/vedtaegter_12.pdf}}. Foreningen blev stiftet i
1974 og har for nyligt holdt 40-års jubilæum. Studenterkøkkenet bliver
vedligeholdt af en frivillig bestyrelse, i samarbejde med brugerne. I lokalet
forefindes der bl.a.:
\begin{itemize}
\itemsep0em
\item Et veludstyret storkøkken.
\item Køleskabe til opbevaring af de studerendes egne mad.
\item Spise/arbejdspladser.
\item Et sofahjørne med projektor.
\end{itemize}

Lokalet bliver stillet til rådighed af universitetet. Kaffe, sodavand, snacks,
mm. kan købes af medlemmer til studentervenlige priser. Foreningen er økonomisk
selvkørende i forhold til det daglige drift; evt. overskud fra salg investeres
i at forbedre studiemiljøet.

Det skal bemærkes at sådan et studenterlokale kun fungerer, hvis de daglige
brugere har et stærkt (fagligt) tilhørsforhold til stedet og hinanden.

\subsection{Daglig brug af studenterkøkkenet}

\begin{itemize}
\itemsep0em 
\item Der stræbes efter arbejdsro fra 8-12 og 13.15-17, og her fungerer lokalet
som læsesal med godt over 100 pladser, som der kan være rift om.
\item I frokostpausen benyttes køkkenfaciliteterne af mange studerende -- f.eks. 
til fælles indkøb/spisning i ``frokostklubber''.
\item Om aftenen samles løse grupper af studerende til madlavning, gerne som
pause fra opgaver.
\end{itemize}

Det skal bemærkes at vi, som datalogistuderende, ikke længere har andre studielokaler.

\subsection{Faste faglige aktiviteter}

I blok 1 2014 blev lokalet blandt andet brugt:
\begin{itemize}
\itemsep0em 
\item tirsdag aftener af (lønnede) mentorer til lektiecafe og madlavning/bespisning af 1. års
bachelorstuderende
\item onsdag aftener til et fælles 1. års kandidat studiegruppe
\item torsdag aftener til madlavning/bespisning af matematikernes ``søsterordning''
\item søndage til brunch/fremvisning af forelæsningsvideoer
\end{itemize}

Lokalets karakter gør, at studerende mødes på tværs af årgange og
fag. At samle folk med stor faglig gejst i et hyggeligt miljø gør,
at frivilligt faglig hjælp og sparring er en del af de studerendes hverdag.

\subsection{Sociale aktiviteter}

Vi vurderer lokalet som værende essentielt for det meste af den
studentersociale aktivitet der foregår ved DIKU. Datalogisk Kantineforening,
som eksempel på en af de mange studenterforeninger der benytter lokalet, afholder f.eks.:

\begin{itemize}
\itemsep0em 
\item Introduktionsarrangement for DIKU bachelorstuderende.
\item Introduktionsarrangement for DIKU kandidatstuderende.
\item Juleklippeklistredag.
\item Julefrokost for DIKU-studerende og modige ansatte.
\end{itemize}

Lokalet, og især køkkenet, bruges også af grupper ved andre studier; bl.a. kan
der nævnes Fysik-, MolBioKem-, Matematik- og  Biologirevyerne (samt
DIKUrevyen), den tværfaglige fredagsbar Cafeen?, og Matematikernes
``søsterordning''. Studenterkøkkenet bruges helt generelt af mange grupper til
madlavning i forbindelse med diverse storfester mm.

Lokalet er også tidligere blevet brugt af DIKU til poster sessions ved konferencer,
julefrokoster, receptioner og deslige.

Vi vurderer desuden studenterkøkkenets miljø som værende en absolut essentiel
forudsætning for mange (datalogi)studerendes studiegennemførsel; især de mere
socialt udsatte bliver der værnet om:

\begin{itemize}
\itemsep0em 
\item Faglig og social støtte tilgængelig stort set døgnet rundt.
\item Ældre studerende fungerer som uformelle mentorer for de yngre, hvilket
specielt er vigtigt for minoriteterne -- f.eks.  kvinderne her på datalogi. 
\end{itemize}

\section{Forhold som DIKU-studerende}

I løbet af de sidste 6 år er Datalogisk Institut gradvist blevet flyttet ud af
bygningen ved Universitetsparken 1. I første omgang angiveligt fordi bygningen
skulle rømmes, men derefter flyttede andre afdelinger fra bl.a. SCIENCE ind i
bygningen.

\subsection{Fysiske lokationer}

DIKU som institut er nu spredt ud over 4-5 lokationer\footnote{APL-gruppen: H.C. Ørsteds Bygningen, Billedegruppen:
Sigurdsgade, HCC-gruppen: Njalsgade (Søndre Campus), Studenterkøkkenet:
Universitetsparken 1, 2. sal, DIKUs Kommunikationsafdeling: Universitetsparken
1, 0. sal}.
\begin{itemize}
\itemsep0em 
\item Dette har haft enorm stor betydning for studerende ved DIKU -- 
kommunikation og samarbejde med de ansatte er blevet væsentligt forringet.
\item Fagligheden og den datalogiske tværfaglighed er blevet meget påvirket af,
at de forskellige forskningssektioner ligger på hver deres addresse.
\item Vi bekymrer os om fremtiden af det enormt velfungerende studiemiljø ved
SCIENCE, Nørre Campus, i tilfælde af endnu en flytning -- tidligere fremsatte
forslag har f.eks. ikke kunne rumme madlavning til/bespisning af de ovennævnte
revyer.
\end{itemize}
Der er nu brug for fremadrettet stabilitet og sikkerhed.

Vi ønsker desuden at gøre opmærksom på, at den tidligere bevaringsværdige
bygning ved Universitetsparken 1 er, grundet misvedligehold, i dårlig stand.
Den fortsat bevaringsværdige Pharmaschool bygning på den anden side af vejen,
som formodenlig er af samme arkitekt, er i væsentligt bedre stand.

I studenterkøkkenet har vi f.eks. i mange år måtte holde et af vinduerne lukket
med en teske i spænd og noget gaffatape. Der har været ``festivalhegn'' omkring
bygningen siden 2008.

\subsection{IT-faciliteter}

\textit{Datalogisk} Instituts EDB-afdeling blev, i forbindelse med konsolidering af IT-infrastruktur på SCIENCE, nedlagt for
nogle år siden.
\begin{itemize}
\itemsep0em 
\item DIKUs EDB-afdeling var primært drevet af deltidsstudentermedhjælpere fra datalogistudiet --
skræddersyede datalogiske løsninger til studenterløn, udført af spirende talenter 
indenfor feltet.
\item Den fungerede tildels som en slags uddannelse for systemadministratorer, uden
at udløse ECTS eller STÅ i sig selv, men som enormt studie- og erhvervsrelevant
fritidsjob.
\item Det skabte nærkontakt mellem ``tjenesteudbyderen'', de studerende og de
ansatte, til fælles faglig gavn -- serviceniveauet var derefter.
\item De løsninger, som Koncern-IT/SCIENCE-IT tilbyder, lever tit ikke op til
de krav, vi som studerende har på et Datalogisk Institut -- især indenfor
vores egne forskningsfelter.
\item Service- og kontaktniveau er heller ikke optimalt, i hvert fald fra de
studerendes synspunkt.
\item Der skal nu bruges undervisningstid til disse datalogi-essentielle
kundskaber, i stedet for uformel kontakt med medstuderende.
\end{itemize}
Situationen er til tider frustrerende som fagmænd m/k.

Vi ærger os egentligt lidt over, at datalogernes faglige gejst og indblik
ikke bliver bedre udnyttet til at skabe datalogiske løsninger til gavn for
Københavns Universitet som helhed.

\subsection{Medindflydelse}

Blandt andet grundet de ovennævnte forhold, frygter vi, at der kan være strukturelle
problemer med hvordan SCIENCE som fakultet administreres, og kommunikationen
med ledelsen generelt. I forhold til den igangværende høringsproces, kan der eksempelvis nævnes:

\begin{itemize}
\itemsep0em 
\item Det er uoverskueligt at sætte os ind i de forhold der vedrører os, grundet den
overvældende arbejdsbyrde i at erhverve dokumentation og tal; tid vi ikke har,
især efter fremdriftsreformens indførsel.
\item Dette stiller os i en urimelig position i forhold til at modarbejde påstande
fremført uden dokumentation.
\end{itemize}

I det hele taget er den bureaukratiske arbejdsbyrde ift. (faglig) udbytte for
den enkelte blevet større og større i løbet af de sidste år, både personligt og
i foreningsarbejde.

\section{Konklusion}

Vi ønsker en løsning på ovennævnte problemer, da vi mener at de sociale og
faglige forhold på SCIENCE, og DIKU især, lider unødvendigt. Administrative
forhold burde ikke kunne være så stor en hæmsko for et velfungerende og
voksende institut, og dertilhørende studiemiljø, der også støtter andre
studieretningers faglige og sociale forhold.

Vores delløsningsforslag ville være at DIKU igen samles omkring
studenterkøkkenet, for at sikre tryghed og stabilitet for den nærmeste
fremtid.

Vi ønsker slutteligt at understrege, at Datalogisk Kantineforening er en
apolitisk forening.
\vspace*{0.5cm}

Datalogisk Fagråd og

Kantinebestyrelsen ved Datalogisk Kantineforening\\

\end{document}

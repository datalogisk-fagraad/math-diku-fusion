\documentclass[a4paper]{article}

\usepackage[utf8]{inputenc}
\usepackage[danish]{babel}

\usepackage{mathpazo}

\usepackage{fancyhdr}%
\usepackage{lastpage}%
%
\pagestyle{fancy}%
\lhead{}%
\chead{}%
\rhead{}%
\cfoot{\thepage/\pageref{LastPage}}%
\renewcommand{\headrulewidth}{0in}%
%

\setlength{\parskip}{0.2cm}
\setlength{\parindent}{0cm}

%\usepackage{draftwatermark}
%\SetWatermarkText{DRAFT}
%\SetWatermarkScale{1}

\begin{document}

ÅBENT BREV

Dekan John Renner Hansen \\
Det Natur- og Biovidenskabelige Fakultet \\
Københavns Universitet

Undertegnede sociale foreninger ved SCIENCE på Nørre Campus (NC) udtrykker
hermed stor bekymring for fremtiden for det sociale og faglige miljø ved
SCIENCE (NC), ifm. den foreslåede fusion af Datalogisk Institut (DIKU) og
Institut for Matematiske Fag (MATH).

Det blev fremlagt til et stormøde den 16. september, at et fusioneret institut
ville blive huset i H.C. Ørsted-bygningerne. Det blev klart meddelt, at det
ikke er ønsket at beholde ``DIKU-bygningen'' grundet ``spildte kvadratmeter'',
og at ``den er dyr som bare pokker at varme op''.

Da Niels Bohr Bygninen også først står klar i 2017, vækker det stor
bekymring for om hvorvidt der fortsat skulle være plads og faciliteter
til det levende sociale og faglige miljø ved SCIENCE (NC).

DIKU-bygningen (Universitetsparken 1) huser bl.a.:

\begin{itemize}

\item Et \emph{studenterdrevet studenterkøkken}.

  Studenterkøkkenet bliver flittigt brugt af studerende fra mange
  uddannelser som et studie-, spise-, og opholdslokale. En bred vifte
  af sociale og faglige foreninger fra hele SCIENCE (NC), benytter
  desuden lokalet til fælles madlavning og bespisning af store
  grupper.

  Lokalets karakter, lokation, og størrelse muliggør
  studieintroduktioner, lektiecaféer, workshops ved konferencer,
  receptioner, julefrokoster, ugelange øvelser op til
  studenterrevyerne, og mange andre faglige og sociale arrangementer.

\item Et \emph{stundenterdrevet kælderlokale}.

  Lokalet fungerer som et øvelokale, depotrum, og produktionsværksted
  for studenterrevyerne, m. fl.

\item Et \emph{stort auditorium, egnet til studenterevyer}.

  Samtlige studenterrevyer ved SCIENCE (NC) afholder deres revyer i
  det store auditorium i DIKU-bygningen. Inventar og erfaringer er
  gennem mange år blevet opbygget omkring, at studenterrevyerne
  afholdes netop her. Auditoriets indretning gør det også enestående
  velegnet til at huse studenterrevyer.

  Kælderetagen huser revyernes fælles rekvisitrum, og kælderen bruges desuden
  til omklædning og rekvisitfremstilling mm. under øvelsesuger og
  forestillinger.

\item En \emph{aflukket øvelsesgang}.

Lokalerne på øvelsesgangen understøtter lektiecaféer og fungerer som
øvelokaler og produktionsværksteder for studenterrevyerne,
studieintroduktioner, mm.

\end{itemize}

Det, at disse faciliteter er samlet i DIKU-bygningen, bidrager i høj
grad til at vores velfungerende faglige og sociale miljøer kan
eksistere; ikke bare på de enkelte institutter, men ved SCIENCE på
hele Nørre Campus.

Vi tilkendegiver hermed, i fællesskab, en \emph{kraftig modstand} imod
alle forringelser af ovennævnte forhold, uden at der er klarhed for
hvorledes de genskabes andetsteds, med \emph{samme kvadratmetertal,
  faciliteter, og relativ beliggenhed}.

Tværtimod ønsker vi at, forholdene i DIKU-bygningen forbedres, da
lokalerne bærer præg af mange års forsømt vedligeholdelse fra
universitetets side.

Slutteligt ønsker vi barmhjertigt at, instituttet, som har skabt de gode
sociale forhold, igen samles under et tag, netop i DIKU-bygningen, for at sikre
at de sociale forhold bliver værnet om fremover.

\begin{flushright}

\footnotesize\sffamily\itshape

Datalogisk Fagråd \\
--

SATYRrevy 2014 \\
-- den fælles studenterrevy for DIKU, MATH, Fysik, Biologi, og MolBioKem i 2014

DIKUrevy \\
-- afholder den årlige studenterrevy på DIKU (grundlagt i 1973)

Matematik Revyen \\
-- afholder den årlige studenterrevy på MATH

FysikRevy \\
-- afholder den årlige studenterrevy på Fysik

Biorevy \\
-- afholder den årlige studenterrevy på Biologi

MBK Revy \\
-- afholder den årlige studenterrevy på Biokemi og molekylær biomedicin

Datalogisk Kantineforening \\
-- driver studenterkøkkenet i Universitetsparken 1 (grundlagt i 1974)

Ruskursusgruppen @ DIKU \\
-- står for studieintroduktionen i samarbejde med instituttet

DIKULAN \\
-- afholder LAN-party 2 gange årligt

Pwnies \\
-- IT-sikkerhed interessegruppe

DQBrew \\
-- DIKU's bryggerlaug

HoTTies \\
-- Typeteori, kategoriteori, og programmeringssprogs interessegruppe

DICON \\
-- afholder brætspilsdag 2 gange årligt

FCSparC \\
-- Fodboldklub v/DIKU

DIKUski \\
-- arrangerer den årlige skitur v/DIKU

DIKU anime \\
-- anime interessegruppe

Harlem klub \\
-- interessegruppe for filmhåndværk;  laver film året rundt

\end{flushright}

\end{document}
